\documentclass[utf8,14pt,a4paper,oneside,russian]{book}
\usepackage[14pt]{extsizes}
\usepackage{longtable}

%===========
%=Кодировка=
%===========
\usepackage[T2A]{fontenc}
\usepackage[utf8]{inputenc}
\usepackage[main=russian, english]{babel}

%===============
%=Римские цифры=
%===============
\newcommand{\RNumb}[1]{\uppercase\expandafter{\romannumeral #1\relax}}

%===================
%=Разметка страницы=
%===================
\usepackage[left=3cm, right=1cm, top=2cm, bottom=2cm, headheight=14pt, headsep=1cm, footskip=1cm]{geometry}
\pagestyle{plain}
\linespread{1.1} %Межстрочный интервад
\setlength{\parindent}{1.25cm} %Абзацный отступ
\setlength{\parskip}{0em} %Интервал между абзацами
\usepackage{indentfirst}

\usepackage{misccorr}
\usepackage{graphicx}
\usepackage{amsmath, amssymb, amsfonts}
\usepackage{xcolor} %пакет для работы с цветом
\usepackage{colortbl} %пакет для работы с цветом в таблицах

%===========
%=Заголовки=
%===========
\usepackage{titlesec}

\titleformat{\section}{\centering\large\bfseries}{\thesection.}{0.5em}{\MakeUppercase}
\renewcommand{\thesection}{\arabic{section}}
\renewcommand{\sectionmark}[1]{\markright{\thesection.~#1}}
\titlespacing{\section}{0em}{2em}{1em}

\titleformat{\subsection}{\centering\normalsize\bfseries}{\thesubsection.}{0.5em}{}
\renewcommand{\thesubsection}{\arabic{section}.\arabic{subsection}}
\titlespacing{\subsection}{0em}{1.25em}{0.5em}

\titlespacing{\paragraph}{0em}{1.05em}{0.25em}

%=======
%=Языки=
%=======
\usepackage{algorithm} %пакеты для работы с алгоритмами
\usepackage{algpseudocode}

\usepackage{listings}

\usepackage{listings}
\lstdefinelanguage
{Assembler}
{morekeywords={CDQE,CQO,CMPSQ,CMPXCHG16B,JRCXZ,LODSQ,MOVSXD, %
		POPFQ,PUSHFQ,SCASQ,STOSQ,IRETQ,RDTSCP,SWAPGS, %
		rax,rdx,rcx,rbx,rsi,rdi,rsp,rbp, %
		r8,r8d,r8w,r8b,r9,r9d,r9w,r9b, %
		r10,r10d,r10w,r10b,r11,r11d,r11w,r11b, %
		r12,r12d,r12w,r12b,r13,r13d,r13w,r13b, %
		r14,r14d,r14w,r14b,r15,r15d,r15w,r15b, mov, times, bits, xor, int, cmp, dw, jmp, jz, inc, db}} % etc.

\lstloadlanguages{C, Python, Assembler, Java}

\definecolor{codegray}{rgb}{0.5,0.5,0.5}
\definecolor{gr}{rgb}{0.97, 0.97, 0.97}

\lstset{language=Assembler,
	extendedchars=true,
	belowcaptionskip=5pt,
	backgroundcolor=\color{gr},
	basicstyle=\ttfamily\normalsize,
	breakatwhitespace=false,         
	breaklines=true, 
	%escapechar=|,
	%frame=tb,
	numbers=left,                          
	numberstyle=\small\color{codegray},         
	stepnumber=1,                         
	numbersep=5pt,
	commentstyle=\itshape,
	stringstyle=\bfseries,
	keepspaces=true,
	showstringspaces=false,
	tabsize=2,
}

%============
%=Содержание=
%============
\usepackage{titletoc}
\makeatletter
\renewcommand{\tableofcontents}{\section{Содержание}\markboth{Содержание}{}\@starttoc{toc}\newpage}
\makeatother
\titlecontents{section}[1.5em]{}{\contentslabel[\thecontentslabel.]{1.5em}}{}{\rule{0.1cm}{0pt}\titlerule[0.75pc]{.}\contentspage}
\titlecontents{subsection}[4em]{\vspace{0.05em}}{\contentslabel[\thecontentslabel.]{2em}}{}{\rule{0.1cm}{0pt}\titlerule[0.75pc]{.}\contentspage}


\setcounter{tocdepth}{3}
\setcounter{secnumdepth}{3}

%========
%=Списки=
%========
\usepackage{enumitem}
\makeatletter
\AddEnumerateCounter{\asbuk}{\@asbuk}{м)}
\makeatother
\setlist{nolistsep, topsep=0.375em}
\setenumerate{leftmargin=2cm, labelsep=0em, labelwidth=0.75cm, align=left}
\setitemize{leftmargin=2cm, labelsep=0em, labelwidth=0.75cm, align=left}
\renewcommand{\labelitemi}{--}
\renewcommand{\labelenumi}{\arabic{enumi})}
\renewcommand{\labelenumii}{\asbuk{enumii})}
\renewcommand{\labelenumiii}{--}

%==========================
%=Математические операторы=
%==========================
\DeclareMathOperator{\mob}{Mob}
\DeclareMathOperator{\fix}{Fix}
\DeclareMathOperator{\ord}{ord}

\graphicspath{{pictures/}}
\DeclareGraphicsExtensions{.pdf,.png,.jpg}

\usepackage{tikz}
\usetikzlibrary{positioning}    % for graph
\usetikzlibrary{graphs}         % for graph

\usepackage{neuralnetwork}

% гиперссылки
\usepackage[hidelinks]{hyperref}

%заглавные римские цифры
\newcommand{\RomanNumeralCaps}[1]
{\MakeUppercase{\romannumeral #1}}

% текстовый филлер для заполнения пропущенных глав
\usepackage{lipsum}  

% двухколоночный оператор
\usepackage{multicol}

% операнд для поворота таблиц
\usepackage{rotating}

%=======================
%=Дополнительные пакеты=
%=======================
\usepackage{geometry}
\usepackage{multirow}
\usepackage{array}
\usepackage{caption} 
\usepackage{makecell}

\begin{document}

\thispagestyle{empty}
\small
\begin{center}
    \includegraphics[width=4.55cm]{logo_mirea}\\
    \MakeUppercase{Минобрнауки России}\\[1em]
    Федеральное государственное бюджетное образовательное учреждение\\
    высшего образования\\[0.5em]
    \textbf{<<МИРЭА -- Российский технологический университет>>}\\
    \textbf{РТУ МИРЭА}\\
    \rule{\textwidth}{0.75pt}\\
    Институт искусственного интеллекта\\
    Базовая кафедра №252 -- информационной безопасности\\[-0.45em]
    \rule{\textwidth}{0.75pt}\\[5em]
    \normalsize\MakeUppercase{\textbf{Выпускная Квалификационная Работа}}\small\\[0.5em]
    Тема:\\ \textbf{<<Устройство и анализ алгоритмов по обходу DPI>>} \\[3em]
    \begin{tabular}{p{7cm}p{6cm}c}
        Студент группы ККСО-03-19    & Николенко В.О.                                      & \rule{2cm}{0.75pt}                    \\[-0.5em]
                                     &                                                     & \footnotesize\textit{(подпись)}\small \\[1em]
        Научниый руководитель        & Жанкевич А.О.                                       & \rule{2cm}{0.75pt}                    \\[-0.5em]
                                     &                                                     & \footnotesize\textit{(подпись)}\small \\[5em]
        Работа представлена к защите & <<\rule{0.5cm}{0.75pt}>> \rule{2cm}{0.75pt} 2023 г. &                                       \\[1em]
        Оценка:                      & <<\rule{0.5cm}{0.75pt}>>                            &                                       \\[1em]
    \end{tabular}
    \vfill
    Москва -- 2024
\end{center}
\normalsize
\newpage

% Содержание
\tableofcontents

% \begin{multicols}{2}
% \end{multicols}

% =================
% === Глоссарий ===
% =================

\newpage
\section{Глоссарий}

Давайте однозначно определим терминологию данной работы, с целью исключить недопонимания между
её авторами и читателями.

\textbf{\textit{DPI}} (англ. Deep Packet Inspection "глубокий анализ пакетов") - технология проверки сетевых пакетов по их содержимому
с целью регулирования и фильтрации трафика, а также накопления статистических данных. В отличие от брандмауэров, DPI
анализирует не только заголовки пакетов, но и полезную нагрузку, начиная со второго уровня модели OSI (канальный).

\textbf{\textit{OSI}} (англ. Open Systems Interconnection "межсетевое взаимодействие открытых систем") - абстрактная модель представления
одних и тех же данных, с которыми можно работать на разных уровнях, предлагаемых данной моделью: физическом, канальном,
сетевом, транспортном, сеансовом, уровне представления, прикладном.

\textbf{\textit{Физический Уровень}} - передача необработанных битов по физическим носителям (кабели, радиосигналы).\\
\textit{Примеры протоколов:} Ethernet (физический аспект) USB, Bluetooth.

\textbf{\textit{Канальный Уровень}} - организация надежной передачи данных между узлами одной сети; управление доступом к среде,
обнаружение ошибок.\\
\textit{Примеры протоколов:} Ethernet (кадры), Wi-Fi (IEEE 802.11), PPP (Point-to-Point Protocol).

\textbf{\textit{Сетевой Уровень}} - маршрутизация и передача пакетов данных между различными сетями.\\
\textit{Примеры протоколов:} IP (Internet Protocol), ICMP (Internet Control Message Protocol), IPSec (для обеспечения безопасности).

\textbf{\textit{Транспортный Уровень}} - обеспечение надежной передачи данных, контроль ошибок, восстановление соединений (TCP/UDP).\\
\textit{Примеры протоколов:} TCP (Transmission Control Protocol), UDP (User Datagram Protocol), SCTP (Stream Control Transmission Protocol).

\textbf{\textit{Сеансовый Уровень}} - управление сессиями, установление, поддержание и завершение сеансов связи между приложениями.\\
\textit{Примеры протоколов:} NetBIOS, PPTP (Point-to-Point Tunneling Protocol), RPC (Remote Procedure Call).

\textbf{\textit{Уровень Представления}} - преобразование данных для представления их в удобном для приложений формате (шифрование, сжатие).\\
\textit{Примеры протоколов:} SSL/TLS (для шифрования), JPEG, PNG (для изображения), ASCII, EBCDIC (кодировки данных).

\textbf{\textit{Прикладной Уровень}} - взаимодействие с конечным пользователем через приложения (веб-браузеры, почтовые клиенты и т. д.).\\
\textit{Примеры протоколов:} HTTP/HTTPS (веб-приложения), FTP (передача файлов), SMTP (электронная почта), DNS (система доменных имен).

\textbf{\textit{Шифрование}} - процесс преобразования данных \textit{(посредством таких операций как: линейных и нелинейных преобразований,
    наложения гаммы, и матричных преобразовний)} в форму, непонятную без ключа. Используется для обеспечения конфиденциальности
информации (например, в коммуникациях).

\textbf{\textit{Дешифрование}} - обратный процесс к шифрованию, преобразующий зашифрованные данные обратно в исходный вид с помощью ключа.

\textbf{\textit{Кодирование}} - преобразование данных в другой формат для удобства передачи или хранения (например, Base64 для передачи
бинарных данных через текстовые протоколы). Кодирование не является средством защиты.

\textbf{\textit{Хеширование}} - одностороннее преобразование данных в уникальную фиксированную строку (хеш), которое невозможно обратным
образом восстановить. Используется для проверки целостности данных, паролей и цифровых подписей.

\textbf{\textit{Прокси-сервер}} - промежуточный сервер в компьютерных сетях, выполняющий роль посредника между пользователем и целевым сервером,
позволяющий клиентам как выполнять косвенные запросы к другим сетевым службам, так и получать ответы.

% ================
% === Введение ===
% ================
\newpage
\section{Введение}

В наше время информационные технологии развиваются уже далеко не семимильными шагами, с каждым годом
появляется все больше методов передачи, шифрования, кодирования, анализа трафика, такие технологии как:
TCP, UDP, TSL, SSL, HTTP, HTTPS, различные подписи с использованием эллиптических кривых, RSA, различные
способы распределения ключей между клиентом и сервером.

Так как трафик это общение не только между машинами, но и между людьми\dots

% ===============
% === 1 глава ===
% ===============

\newpage
\section{Глубокий Анализ Пакетов}

Глубокий анализ пакетов/трафика - основообразующая система для контроля доступа к запрещённым ресурсам на территории конкретного госудраства.
Таким образом достигается либо замедление, либо полная блокировка конкретного ресура. В данный момент, лидером в области ограничения доступа
через национальных провайдеров к неугодным ресурсам является Китай.

\subsection{Подходы к обходу блокировок}

Существует несколько методологий и подходов к обходу блокировок, предлагаю перечислить их по ранжиру, от самого популярного к самым незнаменитым:

\begin{enumerate}
    \item изменение характера трафика
          \begin{itemize}
              \item шифрование трафика (VPN, SSH, TLS)
              \item прокси-сервисы (HTTP/HTTPS-прокси, SOCKS5-прокси)
          \end{itemize}
    \item изменение сигнатур протоколов
          \begin{itemize}
              \item замена сигнатур или же обфускация (Obfsproxy, Meek)
              \item обфускация сессий (шумовой трафик)
          \end{itemize}
    \item обход блокировки на уровне IP и DNS
          \begin{itemize}
              \item изменение DNS (DNS-over-HTTPS/DNS-over-TLS)
              \item TOR (The Onion Router)
              \item I2P (Invisible Internet Project)
          \end{itemize}
    \item туннелирование и псевдотуннелирование
          \begin{itemize}
              \item ICMP - использование протокола ICMP (обычно для ping) для передачи данных.
              \item DNS - использование DNS - запросов для передачи данных (например, через запросы к доменным именам).
              \item GRE - использование протокола GRE для создания виртуальных частных сетей.
          \end{itemize}
    \item маскирование через разрешённые протоколы
          \begin{itemize}
              \item VPN через разрешённые протоколы (OpenVPN over TCP / 443, WireGuard)
              \item туннелирование через WebSocket
          \end{itemize}
    \item  использование распределённых сетей и децентрализованных решений
          \begin{itemize}
              \item decentralized VPN (dVPN)
              \item технологии блокчейн (Orchid, Mysterium)
          \end{itemize}
\end{enumerate}

\subsection{Известные протоколы для доступа к заблокированным ресурсам}

\subsubsection{Shadowsocks}
Shadowsocks - такой протокол который предполагает использование прокси-сервера. Сам же сервер использует шифрование для получения
с клиента и отправки на него данных. Он был разработан для обхода интернет-цензуры и обеспечивает высокую скорость передачи данных.

Авторы взяли классический SOCKS-протокол, который передает все данные в открытом виде и потому очень легко определяется в DPI протоколах,
применили поверх него шифрование с помощью разных алгоритмов, убрали излишний функционал (к примеру, была убрана авторизация в протоколое
по логину и паролю, она стала проводиться по ключу шифрования), и добавили несколько других нововведений по обфускации трафика. И это
сработало - долгое время Shadowsocks был излюбленным инструментом тысяч людей, позволяющим пробиваться через китайский firewall.

\textit{Особенности данного алгоритма:}
\begin{enumerate}
    \item Шифрование: Shadowsocks использует различные алгоритмы шифрования (например, AES-256-GCM), что делает трафик менее заметным для DPI.
    \item Прокси: Это SOCKS5-прокси, который позволяет передавать трафик через сервер, скрывая реальный IP-адрес пользователя.
    \item Обфускация: Некоторые реализации Shadowsocks включают обфускацию трафика, чтобы он выглядел как обычный HTTPS-трафик.
\end{enumerate}

% \begin{tikzpicture}
%     %% Опция nodes определяет параметры всех узлов графа. align=center центрирует текст в узле
%     \graph[nodes={align=center}, grow down sep, branch right sep] {
%     Диакритика? ->
%     {
%     "Много над E:\\ \`{E}, \'{E}, \^{E}, \"{E}" ->  Французский,
%     Мало -> "Хоть что-то есть?" ->
%     {
%     "\c{C} и \"{E}" -> Точно не французский? ->
%     {
%     "Ой$\dots$он" -> Французский,
%     Вроде нет -> Албанский
%     },
%     "Только \"{A} и \"{O} \\ но мноогоо \\ сдвооенных" -> Финский
%     }
%     }
%     };
% \end{tikzpicture}

Хронологическое дерево алгоритмов можно представить в виде графа:

\begin{tikzpicture}
    %% Опция nodes определяет параметры всех узлов графа. align=center центрирует текст в узле
    \graph[nodes={align=center}, grow down sep=1cm, grow down sep, branch right sep, branch left sep] {
    SOCKS -> Shadowsocks -> V2Ray -> V2Fly -> XRay ->
    {
    VMess,
    VLESS,
    XTLS,
    },
    };
\end{tikzpicture}

\subsubsection{VMess}
Непосредственно классических протоколов в V2Ray и XRay всего два не считая VLite: VMess, VLESS.

VMess - протокол, разработанный для использования с V2Ray. Это более сложная и мощная система, чем Shadowsocks. Протокол предлагает больше
функций для обхода цензуры. Поддерживат определение "свой/чужой" по ID пользователя и опционально шифрование данных.

На данный момент VMess считается устаревшим, а при работе через TCP - небезопасным, однако вариант VMess-over-Websockets-over-TLS\\
по-прежнему вполне себе жизнеспособен и может использоваться при отсуствии поддерживаемых в каком-либо клиенте альтернатив.

\textit{Особенности данного алгоритма:}
\begin{enumerate}
    \item Шифрование: VMess использует шифрование как для данных, так и для заголовков, что делает его более защищенным от анализа.
    \item Динамическое изменение портов: Протокол может динамически изменять порты и другие параметры для затруднения распознавания.
    \item Мультиплексирование: VMess поддерживает мультиплексирование соединений, что позволяет экономить ресурсы и улучшать производительность.
\end{enumerate}


\subsubsection{VLESS}
VLESS — новая и улучшенная версия VMess, которая была разработана с акцентом на производительность и безопасность.
Протокол предлагает упрощенный подход к шифрованию и аутентификации. В отличие от VMess он не предусматривает механизма шифрования
(предполагается, что будет использоваться стандартное шифрование по типу TLS), а только проверку "свой/чужой" и "паддинг" данных
(изменение размеров пакетов для затруднения узнавания паттернов трафика). В протоколе исправлен ряд уязвимостей старого VMess,
и он активно развивается - например, автор планирует добавить поддержку компрессии алгоритмом Zstandard, не столько для производительности,
сколько для затруднения анализа "снаружи".

\textit{Особенности данного алгоритма:}
\begin{enumerate}
    \item Безопасность: VLESS не использует шифрование заголовков, что делает его более легковесным и быстрым.
    \item Поддержка различных транспортных протоколов: VLESS может работать через WebSocket, gRPC и другие транспортные протоколы, что помогает обойти блокировки.
    \item Аутентификация: Использует различные методы аутентификации, включая UUID, что упрощает настройку.
\end{enumerate}

% ============================
% === 6 глава - Заключение ===
% ============================

\newpage
\section{Заключение}


% ===================================
% === 7 глава - Список литературы ===
% ===================================

\newpage
\section{Список литературы}

\end{document}